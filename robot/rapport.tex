\documentclass{scrartcl}
\usepackage[utf8]{inputenc}
\usepackage[frenchb]{babel}
\usepackage{hyperref}

\author{Maxence Ahlouche \and Maxime Arthaud \and Korantin Auguste
  \and Martin Carton \and Thomas Forgione \and Thomas Wagner}

\title{Rapport du projet d'ingénierie robotique}
\date{9 décembre 2013}

\begin{document}
\maketitle
\tableofcontents
\newpage

\section{Présentation de l'équipe}
Cette équipe a été menée par Maxime Arthaud, assisté de son Responsable
Qualité Thomas Forgione. Les autres membres de l'équipe sont Martin Carton,
Maxence Ahlouche, Korantin Auguste et Thomas Wagner.

\section{Méthodes de programmation du robot}
\subsection{Programmation graphique}

Au début du projet, nous avons commencé à programmer le robot comme
indiqué dans le sujet, c'est à dire en utilisant le logiciel de programmation
graphique disponible sur Windows. Seulement, ce logiciel, bien
qu'accessible à tout le monde, nous a semblé très peu pratique % pas à tout le monde, mais à ceux qui paye une licence pour un mauvais OS
à utiliser, si bien que nous n'avons pas été très productif lors des
premières séances.

En effet, dès que nous souhaitions faire autre
chose que des instructions de base (par exemple, faire tourner le
moteur à une vitesse dépendant de la valeur d'un capteur), nous
passions beaucoup trop de temps à essayer de comprendre comment fonctionnait
l'interface.

De plus, nous avons rencontré un problème que nous n'avons pas réussi
à résoudre: notre robot, avançait de manière saccadée. Bien que nous ayons
quelques hypothèses sur l'origine de ce problème (bug ou documentation non à
jour), nous n'avons pas réussi à le résoudre avec le logiciel fourni.

À cause de ces désagréments, nous avons décidé d'essayer
d'autres méthodes pour programmer le robot.

\subsection{Librairie Python}

Un des membres de notre équipe a trouvé sur Internet une librairie en
Python, qui permet de contrôler un robot branché en USB. Cette
librairie (qui s'appelle \texttt{nxt-python}) nous a permis de régler
notre problème d'avancement saccadé.

Elle nous a également permis de programmer notre robot de manière très
simple, et de gagner beaucoup de temps par rapport au logiciel de
programmation graphique.

Elle présente toutefois un inconvénient majeur: elle ne permet pas de télécharger
le programme sur le robot (le python étant interprété sur l'ordinateur);
par conséquent, le robot devait rester branché à l'ordinateur lors de
l'exécution du programme, et nous devions le suivre avec l'ordinateur. Ce qui
n'est évidemment pas pratique.

\subsection{Langage NXC}

Finalement, nous avons décidé de programmer le robot en utilisant le
langage NXC (\emph{Not Exactly C}), développé spécifiquement pour
les robots NXT.

Le compilateur que nous avons trouvé nous permet également de
télécharger le programme sur le robot, donc nous avons résolu le
problème posé par la librairie en Python, ainsi que ceux posés par le
logiciel de programmation graphique.

\section{Suivie de murs}

Nous avons réalisé un programme qui permet au robot de longer un mur, grâce
à NXC.

Nous avons fait le choix, par manque de capteurs, de ne longer que les murs à
gauche du robot.

\subsection{Suivie d'un mur}

Afin de longer un mur, nous avons posé un capteur à ultrasons sur le
côté gauche de notre robot. Lorsque nous détectons que nous sommes
trop éloignés du mur, nous tournons légèrement à gauche.

Cette méthode donnant des résultats peu probants si la direction
initiale du robot n'est pas exactement parallèle au mur, nous avons
décidé d'ajouter un autre capteur à ultrasons à l'avant du robot. Ainsi,
notre robot ne fonce plus dans les murs, et tourne à droite quand le
capteur de devant détecte quelque chose.

Ainsi, nous pouvons suivre un mur droit de manière assez régulière.

\subsection{Lissage du mouvement du robot}

Afin d'éviter d'avoir un mouvement erratique lorsqu'on n'est pas
exactement parallèle à la paroi que l'on doit longer (mouvements de
``rebondissement'' sur le mur, par exemple), nous avons décidé
d'améliorer notre algorithme via une astuce très simple: le changement de
direction dépend de la distance du mur. Ainsi, si nous nous éloignons
légèrement du mur (à cause d'une trajectoire non parallèle au mur, par
exemple), alors nous tournerons très légèrement à gauche, sans
ralentir. Ainsi, nous nous retrouverons presque parallèle au mur.

Dans le cas d'un virage à gauche du mur, le capteur détecte un mur à
l'infini; dans ce cas là, nous tournons autant que possible, jusqu'à
retrouver un mur. Dès que le capteur détecte un mur, on repasse dans
le cas précédent; par conséquent, notre robot ne se retrouve quasiment
jamais face au mur qu'il devait longer. De plus, ceci présente
l'avantage que notre robot est désormais capable de faire demi-tour
autour d'un mur qui s'arrête d'un coup, sans virages (comme dans le
cas des planches fournies par notre professeur pour tester le robot).

Dans le cas d'un virage à droite du mur, notre algorithme ne change
pas.

Ainsi, nous avons un robot capable de longer les murs de manière très
fiable, peu importe l'angle des murs.

\end{document}
