\documentclass{scrartcl}
\usepackage[utf8]{inputenc}
\usepackage[frenchb]{babel}
\usepackage{amssymb}
\usepackage{lmodern}
\usepackage[T1]{fontenc}
\usepackage{hyperref}
\usepackage{verbatim}
\usepackage{listings}
\usepackage{graphicx}

\usepackage{color}
\definecolor{deepblue}{rgb}{0,0,0.5}
\definecolor{deepred}{rgb}{0.6,0,0}
\definecolor{deepgreen}{rgb}{0,0.5,0}

\newcommand{\brokencell}[2][c]{\begin{tabular}[#1]{@{}c@{}}#2\end{tabular}}

\lstset{frame=single, breaklines=true,
          breakatwhitespace=true, basicstyle=\scriptsize,
          showstringspaces=false, escapeinside={(*}{*)},
          keywordstyle=\color{deepblue},
          stringstyle=\color{deepred},
          commentstyle=\color{deepgreen},
          literate=
                   {é}{{\'e}}1{É}{{\'E}}1
                   {è}{{\`e}}1{È}{{\`E}}1
                   {ê}{{\^e}}1{Ê}{{\^E}}1
                   {à}{{\`a}}1{À}{{\`A}}1
                   {ù}{{\`u}}1{Ù}{{\`U}}1
                   {û}{{\^u}}1{Û}{{\^U}}1
                   {ô}{{\^o}}1{Ô}{{\^O}}1
                   {ó}{{\'o}}1{Ó}{{\'O}}1
                   {ç}{{\c c}}1{Ç}{{\c C}}1
                   {œ}{{\oe}}1{Œ}{{\OE}}1
        }

\begin{document}
\title{Rapport du projet de théorie des jeux}
\author{Maxence Ahlouche \and Maxime Arthaud \and Korantin Auguste
          \and Martin Carton \and Thomas Forgione \and Thomas Wagner}
\date{11 novembre 2013}
\maketitle
\tableofcontents
\lstlistoflistings
\newpage

\section{Présentation de l'équipe}
  Cette équipe a été menée par Thomas Forgione, assisté de son Responsable
  Qualité Maxence Ahlouche. Les autres membres de l'équipe sont Martin Carton,
  Thomas Wagner, Maxime Arthaud, et Korantin Auguste.

  Tous les membres de l'équipe ont été présents à chacune des séances lors de
  cette UA.

\section{Shifumi}

    Une stratégie simple et efficace à laquelle on pourrait penser pour gagner
    au Shifumi serait de jouer de manière aléatoire.

    Et en effet, il s'avère que si les deux joueurs jouent de manière équiprobable,
    on a affaire à un équilibre de Nash : aucun changement de stratégie de la part
    d'un joueur ne pourra lui permettre d'augmenter ses chances de gagner.

    De plus, si un adversaire ne joue pas de manière aléatoire (ou augmente les probabilités
    de jouer un certain élément), alors on pourra trouver une stratégie efficace contre lui.

    \subsection{Stratégie développée}

    Afin de démontrer qu'un adversaire ne jouant pas aléatoirement est facile
    à battre, nous avons développé une stratégie qui se base sur des chaînes
    de Markov : en se basant sur les N derniers éléments joués, elle regarde
    dans l'historique pour voir l'élément qui était joué le plus souvent par
    l'adversaire.

    Cette stratégie s'avère vraiment efficace contre un joueur humain.

    Toutefois, elle est prévisible : si on sait qu'on à affaire à une telle
    stratégie, on peut jouer de manière à la battre.

    C'est pour cela qu'une stratégie aléatoire est la seule pouvant maximiser
    nos gains dans le pire des cas.

    \subsection{Variantes}

    Toutes les variantes du Shifumi qui consistent à rajouter des éléments pour
    obtenir un nombre d'éléments pair va créer un déséquilibre, car un élément
    sera moins efficace contre les autres. L'équilibre de Nash du jeu va alors
    consister à ne jamais jouer cet élément.

    Si le nombre d'éléments est impair, alors le jeu pourra être équilibré,
    comme un Shifumi classique.

\section{Morpion}
\section{Compétition/Duopole}
  Le but de ce jeu est de maximiser le gain d'une entreprise en concurrence
  avec une autre entreprise en fonction de leur production.

  \subsection{Analyse}
    Notre gain étant égal à $g_x(x) = -x(x+y-3)$ avec $x$ et $y$ les
    productions respectives des deux entreprises, pour le maximiser il
    suffirait de jouer $x=\frac{3-y}{2}$ (c'est que fait la stratégie fournie
    \verb+noncooperatif+).
    
    Cependant au moment de décider quelle quantité produire nous ne connaissons
    pas la production $y$ de l'entreprise concurrente. De plus si les deux
    entreprises s'ignorent totalement en essayant de maximiser leur seul gain,
    elles gagneront au final moins que deux entreprises qui coopèrent
    totalement c'est à dire qui chercheraient à maximiser leur gain total
    $g_x(x)+g_y(y)$ avec $g_x(x) = g_y(y)$ c'est à dire en jouant $x=y=0.75$.
    En effet, elles gagneront alors chacune $\frac 9 8$ à chaque tour au lieu
    de $1$ si elles sont toutes les deux non-coopératives.

    Nous avons donc essayé plusieurs stratégies différentes, qui essayent de
    maximiser le gain de l'entreprise en tenant en compte l'autre entreprise,
    les productions et les gains des tours précédents.

  \subsection{Stratégies}
    \subsubsection{Stackelberg}
      Todo: Pourquoi 2/3? Pourquoi 1.1*2/3 c'est mieux.

      Une variante de cette stratégie consiste à utiliser la production moyenne
      de l'autre joueur plutôt que seulement la dernière valeur. Elle permet
      d'obtenir des résultats légèrement meilleurs.

      De plus en coopérant avec l'autre joueur (voir listing~\ref{lst:mmmttk})
      si celui-ci coopère, on obtient de meilleurs résultats.

      Enfin, une autre variante de cette stratégie (voir
      listing~\ref{lst:mmmttkv2}) maximise le gain si l'autre joueur a joué une
      constante sur les derniers tours. Cette variante donne des résultats
      un peu moins bons en moyenne mais est meilleure dans le meilleur des cas.

    \subsubsection{Stratégie pénalisante}
      % Todo: ...  est d'être coopérati(f|ve) ? agressi(f|ve)
      Le principe de cette stratégie (voir listing~\ref{lst:penalise}) est
      d'être coopératif tant que l'adversaire l'est, et de devenir plus
      agressif quand il ne l'est plus: à chaque fois que l'autre joueur n'est
      pas coopératif, on joue comme le ferait la stratégie Stackelberg.

      Une variante de cette stratégie (voir listing~\ref{lst:penaliseviolent})
      consiste à le pénaliser de plus en plus: la première fois on le pénalise
      une fois, puis deux, puis trois, etc.

      Ces deux stratégies sont efficaces à la fois quand l'autre joueur est
      coopératif (on est alors coopératif) et contre un joueur non-coopératif
      (on devient alors agressif).

    \subsubsection{Stratégie évolutive}
      Une autre stratégie (voir listing~\ref{lst:pokemon}) que nous avons
      développée consiste à augmenter la production si la dernière augmentation
      a augmenté notre gain ou si la dernière diminution l'a diminué et
      vice-versa.

      Celle-ci est plutôt efficace, mais n'est pas la meilleure que nous ayons
      développée: elle se met souvent à osciller inutilement.

    \subsubsection{Stratégie polynomiale} \label{sec:poly}
      Enfin, une autre stratégie (voir listing~\ref{lst:poly}) joue en fonction
      de la production moyenne de l'autre joueur telle que:
      \begin{itemize}
        \item $f(0) = 1.125$: on joue beaucoup si l'autre joue peu, sans jouer
          trop pour ne pas le fâcher;
        \item $f(0.75) = 0.75$: elle coopère avec quelqu'un qui coopère;
        \item $f(1.5) = 0.75$: elle coopère avec quelqu'un qui ne coopère pas,
          pour essayer de faire coopérer celui-ci (c'est dans notre intérêt et
          ça ne changerait pas son gain, il est donc possible qu'elle le
          fasse).
      \end{itemize}

      On choisit alors la fonction $f$ pour être un polynôme qui interpole ces
      valeurs.

      Cette méthode s'avère très efficace en moyenne.

  \subsection{Comparaison}
    Les tables~\ref{table:coop_results} et~\ref{table:coop_results2} montrent
    les résultats obtenus par les quelques stratégies que nous avions à notre
    disposition\footnote{Les stratégies fournies par les professeurs
    (marquées~*), les notre et des stratégies «~prêtées~» par d'autres
    groupes pour les tests (marquées~**) que nous remercions.} pour une durée
    de $99$ tours et $1000$ tours respectivement. Nous faisons s'opposer toutes
    les stratégies entre elles puis notons pour chacune d'elles son gain
    minimal, moyen et maximal\footnote{Cette table peut être générée par le
    script matlab \textit{comp\_tests.m} fourni dans l'archive.}.

    Todo: mettre à jour avec les dernières valeurs quand on aura fini et mettre
    les résultats pour un autre nombre de tours: il suffit d'appeler
    \verb+comp_tests(100, true)+ d'ajouter les * et les $\backslash$\_ et
    supprimer cette phrase quand on aura fini.
    \begin{table}[f]
      \centering
      \begin{tabular}{|c||c|c|c|}
        \hline
        Stratégie      & Gain minimal & Gain moyen & Gain maximum \\\hline\hline
           cooperatif* & $ 55.31$ & $ 97.84$ & $111.38$ \\\hline
        noncooperatif* & $ 42.96$ & $ 87.74$ & $130.26$ \\\hline
          stackelberg* & $ 48.00$ & $ 79.75$ & $114.69$ \\\hline
                palkeo & $ 15.75$ & $ 92.49$ & $111.38$ \\\hline
              penalise & $ 45.53$ & $ 86.45$ & $111.87$ \\\hline
     penalise\_violent & $ 41.87$ & $ 89.08$ & $111.38$ \\\hline
                mmmttk & $ 48.13$ & $ 91.19$ & $111.38$ \\\hline
              mmmttkv2 & $ 48.92$ & $ 78.55$ & $123.19$ \\\hline
              gklmjbse & $ 55.31$ & $ 83.89$ & $116.29$ \\\hline
                  poly & $ 55.53$ & $100.21$ & $111.38$ \\\hline
              killer** & $  0.00$ & $ 71.91$ & $114.64$ \\\hline
     cooperatifmixte** & $ 61.43$ & $ 97.84$ & $111.38$ \\\hline
      agressivemieux** & $  3.15$ & $ 74.76$ & $112.18$ \\\hline
     best\_strategie** & $ 32.11$ & $ 86.15$ & $122.62$ \\\hline
      \end{tabular}
      \caption{Résultats des différentes stratégies sur 100 tours}
      \label{table:coop_results}
    \end{table}
    \begin{table}[f]
      \centering
      \begin{tabular}{|c||c|c|c|}
        \hline
        Stratégie      & Gain minimal & Gain moyen & Gain maximum \\\hline\hline
         cooperatif* & $561.56$ & $978.59$ & $1123.88$ \\\hline
      noncooperatif* & $380.79$ & $877.10$ & $1317.08$ \\\hline
        stackelberg* & $498.00$ & $797.42$ & $1132.44$ \\\hline
              palkeo & $694.54$ & $986.94$ & $1123.88$ \\\hline
            penalise & $419.12$ & $860.12$ & $1124.70$ \\\hline
   penalise\_violent & $421.22$ & $896.10$ & $1123.88$ \\\hline
              mmmttk & $492.11$ & $923.94$ & $1123.88$ \\\hline
            mmmttkv2 & $531.85$ & $800.61$ & $1262.25$ \\\hline
            gklmjbse & $561.56$ & $832.41$ & $1135.33$ \\\hline
                poly & $561.82$ & $1011.24$ & $1123.88$ \\\hline
            killer** & $  0.00$ & $773.86$ & $1133.09$ \\\hline
   cooperatifmixte** & $698.67$ & $990.58$ & $1123.88$ \\\hline
    agressivemieux** & $  3.15$ & $750.64$ & $1126.18$ \\\hline
   best\_strategie** & $322.67$ & $881.00$ & $1262.81$ \\\hline
      \end{tabular}
      \caption{Résultats des différentes stratégies sur 1000 tours}
      \label{table:coop_results2}
    \end{table}

    On remarque que notre meilleure stratégie en moyenne est la stratégie qui
    utilise un polynôme (voir section~\ref{sec:poly}), cette stratégie est
    aussi plutôt bonne dans le pire des cas (elle est donc adaptée à une
    entreprise qui souhaiterait minimiser ses risques).

    Todo: parler de palkeo qui est mauvaise à cours terme et devient bonne à
    long terme dans le pire des cas.

    On remarque aussi que nos stratégie gagneront toujours de l'argent,
    contrairement à la stratégie nommée ``killer'' fournie par un autre groupe.

    Enfin on remarque que la stratégie non-coopérative est plutôt bonne dans le
    pire des cas et en moyenne et est la meilleure dans le meilleur des cas.
    Elle est donc adaptée à une entreprise qui serait prête à prendre quelques
    risques pour avoir une chance de gagner plus.
    
\section{Annexes}
  \lstinputlisting[label=lst:mmmttk, language=matlab, caption=Statégie Stackelberg sur la moyenne]{duopole/mmmttk.m}
  \lstinputlisting[label=lst:mmmttkv2, language=matlab, caption=Statégie Stackelberg sur la moyenne (variante)]{duopole/mmmttkv2.m}
  \lstinputlisting[label=lst:penalise, language=matlab, caption=Statégie pénalisante]{duopole/penalise.m}
  \lstinputlisting[label=lst:penaliseviolent, language=matlab, caption=Statégie pénalisante (variante)]{duopole/penalise_violent.m}
  \lstinputlisting[label=lst:pokemon, language=matlab, caption=Stratégie évolutive]{duopole/evolutif.m}
  \lstinputlisting[label=lst:poly, language=matlab, caption=Stratégie polynomiale]{duopole/poly.m}
  
\end{document}
