\documentclass{scrartcl}
\usepackage[utf8]{inputenc}
\usepackage[frenchb]{babel}
\usepackage{amssymb}
\usepackage{lmodern}
\usepackage[T1]{fontenc}
\usepackage{hyperref}
\usepackage{verbatim}
\usepackage{listings}
\usepackage{graphicx}


\usepackage{color}
\definecolor{deepblue}{rgb}{0,0,0.5}
\definecolor{deepred}{rgb}{0.6,0,0}
\definecolor{deepgreen}{rgb}{0,0.5,0}

\newcommand{\brokencell}[2][c]{\begin{tabular}[#1]{@{}c@{}}#2\end{tabular}}

\lstset{frame=single, breaklines=true,
          breakatwhitespace=true, basicstyle=\scriptsize,
          showstringspaces=false, escapeinside={(*}{*)},
          keywordstyle=\color{deepblue},
          stringstyle=\color{deepred},
          commentstyle=\color{deepgreen},
          literate=
                   {é}{{\'e}}1{É}{{\'E}}1
                   {è}{{\`e}}1{È}{{\`E}}1
                   {ê}{{\^e}}1{Ê}{{\^E}}1
                   {à}{{\`a}}1{À}{{\`A}}1
                   {ù}{{\`u}}1{Ù}{{\`U}}1
                   {û}{{\^u}}1{Û}{{\^U}}1
                   {ô}{{\^o}}1{Ô}{{\^O}}1
                   {ó}{{\'o}}1{Ó}{{\'O}}1
                   {ç}{{\c c}}1{Ç}{{\c C}}1
                   {œ}{{\oe}}1{Œ}{{\OE}}1
        }

\begin{document}
\title{Rapport du projet de programmation linéaire}
\author{Maxence Ahlouche \and Maxime Arthaud \and Korantin Auguste
          \and Martin Carton \and Thomas Forgione \and Thomas Wagner}
\date{11 novembre 2013}
\maketitle
\tableofcontents
\lstlistoflistings
\newpage

\section{Présentation de l'équipe}
  Cette équipe a été menée par Maxence Ahlouche, assisté de son Responsable
  Qualité Thomas Wagner. Les autres membres de l'équipe sont Martin Carton,
  Thomas Forgione, Maxime Arthaud, et Korantin Auguste.

  Todo si nécessaire, torm sinon:
  \begin{table}[h]
    \centering
    \begin{tabular}{|c||c|c|c||c|c|c|}
      \hline
      & TD1 & TD2 & TD3 & TP1 & TP2 & TP3 \\
      \hline\hline
      Maxence Ahlouche (CPC) & & & & & & \\
      \hline
      Maxime Arthaud & & & & & & \\
      \hline
      Korantin Auguste &&&&&& \\
      \hline
      Carton Martin &&&&&&\\
      \hline
      Thomas Forgione &&&&&&\\
      \hline
      Thomas Wagner (RQ) &&&&&&\\
      \hline
    \end{tabular}
  \end{table}

\section{Shifumi}
\section{Morpion}
\section{Compétition/Duopole}
  Le but de ce jeu est de maximiser le gain d'une entreprise en concurrence avec
  une autre entreprise en fonction de la production.

  Notre gain étant égal à $g(x) = -x(x+y-3)$ avec $x$ et $y$ les productions
  respectives des deux entreprises, pour le maximiser il suffirait de jouer
  $x=\frac{3-y}{2}$. Cependant au moment de décider que produire nous ne
  connaissons pas la production $y$ de l'entreprise concurrente.

  Nous avons essayé plusieurs stratégies différentes.

  \subsection{Stratégies}
    todo: les stratégies des profs on les explique ou pas?
    \subsubsection{Coopératif}
    Todo: pourquoi 0.75? Avec $\frac{7-\sqrt{13}}{4}$ on a les mêmes résultats.

    \subsubsection{Stackelberg}
      Todo: Pourquoi 2/3? Pourquoi 1.1*2/3 c'est mieux.

      Une variante de cette stratégie (voir listing~\ref{lst:stackmean})
      consiste à utiliser la production moyenne de l'«~adversaire~» plutôt que
      seulement la dernière valeur. Elle permet d'obtenir des résultats
      légèrement meilleurs.

    \subsubsection{Pénalise}
      Le principe de cette stratégie (voir listing~\ref{lst:penalise}) est
      d'être coopératif tant que l'adversaire l'est, et de devenir plus
      agressif quand il ne l'est plus: à chaque fois que l'«~adversaire~» n'est
      pas coopératif, on joue comme le ferait la stratégie Stackelberg.

      Une variante de cette stratégie (voir listing~\ref{lst:penaliseviolent})
      consiste à le pénaliser de plus en plus: la première fois on le pénalise
      une fois, puis deux, puis trois, etc.

      Ces deux stratégies sont efficaces à la fois quand l'autre joueur est
      coopératif (on est alors coopératif) et contre un joueur non-coopératif
      (on devient alors agressif).

  \subsection{Comparaison}
    La table~\ref{table:coop_results} montre les résultats obtenus par les
    quelques stratégies que nous avions à notre disposition\footnote{Cette
    table peut être générée par le script matlab \textit{comp\_tests.m} fourni
    dans l'archive.}.

    \begin{table}[h]
      \centering
      \begin{tabular}{|c||c|c|c|}
        \hline
        Stratégie      & Gain minimal & Gain moyen & Gain maximum \\\hline\hline
        Coopératif     & $110.75$     & $114.02$   & $125.156$    \\\hline
        Non coopératif & $83.25$      & $96.16$    & $109.13$     \\\hline
        Stackelberg    & $54.42$      & $64.67$    & $72.75$      \\\hline
        Pénalise       & $0$          & $44.98687$ & $109.01$     \\\hline
      \end{tabular}
      \caption{Résultats des différentes stratégies}
      \label{table:coop_results}
    \end{table}
    
\section{Annexes}
  \lstinputlisting[label=lst:stackmean, language=matlab, caption=Statégie Stackelberg sur la moyenne]{duopole/old/stackelberg_mean1.m}
  \lstinputlisting[label=lst:penalise, language=matlab, caption=Statégie pénalisante]{duopole/old/penalise1.m}
  \lstinputlisting[label=lst:penaliseviolent, language=matlab, caption=Statégie pénalisante (variante)]{duopole/old/penalise_violent1.m}
  
\end{document}
