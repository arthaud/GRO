\documentclass{scrartcl}
\usepackage[utf8]{inputenc}
\usepackage[frenchb]{babel}
\usepackage{amssymb}
\usepackage{lmodern}
\usepackage[T1]{fontenc}
\usepackage{hyperref}
\usepackage{verbatim}
\usepackage{listings}
\usepackage{graphicx}


\usepackage{color}
\definecolor{deepblue}{rgb}{0,0,0.5}
\definecolor{deepred}{rgb}{0.6,0,0}
\definecolor{deepgreen}{rgb}{0,0.5,0}

\newcommand{\brokencell}[2][c]{\begin{tabular}[#1]{@{}c@{}}#2\end{tabular}}

\lstset{frame=single, breaklines=true,
          breakatwhitespace=true, basicstyle=\scriptsize,
          showstringspaces=false, escapeinside={(*}{*)},
          keywordstyle=\color{deepblue},
          stringstyle=\color{deepred},
          commentstyle=\color{deepgreen},
          literate=
                   {é}{{\'e}}1{É}{{\'E}}1
                   {è}{{\`e}}1{È}{{\`E}}1
                   {ê}{{\^e}}1{Ê}{{\^E}}1
                   {à}{{\`a}}1{À}{{\`A}}1
                   {ù}{{\`u}}1{Ù}{{\`U}}1
                   {û}{{\^u}}1{Û}{{\^U}}1
                   {ô}{{\^o}}1{Ô}{{\^O}}1
                   {ó}{{\'o}}1{Ó}{{\'O}}1
                   {ç}{{\c c}}1{Ç}{{\c C}}1
                   {œ}{{\oe}}1{Œ}{{\OE}}1
        }

\begin{document}
\title{Rapport du projet de théorie des jeux}
\author{Maxence Ahlouche \and Maxime Arthaud \and Korantin Auguste
          \and Martin Carton \and Thomas Forgione \and Thomas Wagner}
\date{11 novembre 2013}
\maketitle
\tableofcontents
\lstlistoflistings
\newpage

\section{Présentation de l'équipe}
  Cette équipe a été menée par Maxence Ahlouche, assisté de son Responsable
  Qualité Thomas Wagner. Les autres membres de l'équipe sont Martin Carton,
  Thomas Forgione, Maxime Arthaud, et Korantin Auguste.

  Todo si nécessaire, torm sinon:
  \begin{table}[h]
    \centering
    \begin{tabular}{|c||c|c|c||c|c|c|}
      \hline
      & TD1 & TD2 & TD3 & TP1 & TP2 & TP3 \\
      \hline\hline
      Maxence Ahlouche (CPC) & & & & & & \\
      \hline
      Maxime Arthaud & & & & & & \\
      \hline
      Korantin Auguste &&&&&& \\
      \hline
      Carton Martin &&&&&&\\
      \hline
      Thomas Forgione &&&&&&\\
      \hline
      Thomas Wagner (RQ) &&&&&&\\
      \hline
    \end{tabular}
  \end{table}

\section{Shifumi}
\section{Morpion}
\section{Compétition/Duopole}
  Le but de ce jeu est de maximiser le gain d'une entreprise en concurrence avec
  une autre entreprise en fonction de la production.

  Notre gain étant égal à $g(x) = -x(x+y-3)$ avec $x$ et $y$ les productions
  respectives des deux entreprises, pour le maximiser il suffirait de jouer
  $x=\frac{3-y}{2}$. Cependant au moment de décider que produire nous ne
  connaissons pas la production $y$ de l'entreprise concurrente.

  Nous avons essayé plusieurs stratégies différentes.

  \subsection{Stratégies}
    todo: les stratégies des profs on les explique ou pas?
    \subsubsection{Coopératif}
    Todo: pourquoi 0.75? Avec $\frac{7-\sqrt{13}}{4}$ on a les mêmes résultats.

    \subsubsection{Stackelberg}
      Todo: Pourquoi 2/3? Pourquoi 1.1*2/3 c'est mieux.

      Une variante de cette stratégie consiste à utiliser la production moyenne
      de l'autre jour plutôt que seulement la dernière valeur. Elle permet
      d'obtenir des résultats légèrement meilleurs.

      De plus en coopérant avec l'autre joueur (voir listing~\ref{lst:mmmttk})
      si celui-ci coopère, on obtient de meilleurs résultats.

      Enfin, une variante de cette stratégie (voir listing~\ref{lst:mmmttkv2})
      maximise le gain si l'autre joueur a joué une constante sur les derniers
      tours. Cette variante donne des résultats moyens un peu moins bons, mais
      est la meilleure dans le pire des cas: elle est donc adapté à une
      entreprise qui veut minimiser ses risques. Elle est aussi meilleure dans
      le meilleur des cas.

    \subsubsection{Stratégie pénalisante}
      Le principe de cette stratégie (voir listing~\ref{lst:penalise}) est
      d'être coopératif tant que l'adversaire l'est, et de devenir plus
      agressif quand il ne l'est plus: à chaque fois que l'autre joueur n'est
      pas coopératif, on joue comme le ferait la stratégie Stackelberg.

      Une variante de cette stratégie (voir listing~\ref{lst:penaliseviolent})
      consiste à le pénaliser de plus en plus: la première fois on le pénalise
      une fois, puis deux, puis trois, etc.

      Ces deux stratégies sont efficaces à la fois quand l'autre joueur est
      coopératif (on est alors coopératif) et contre un joueur non-coopératif
      (on devient alors agressif).

    \subsubsection{Stratégie évolutive}
      Une autre stratégie que nous avons développée consiste à augmenter la
      production si la dernière augmentation a augmenter notre gain ou si la
      dernière diminution l'a diminué et vice-versa.

      Celle-ci est plutôt efficace, mais n'est pas la meilleure que nous ayons
      développée: elle se met souvent à cycler inutilement.

  \subsection{Comparaison}
    La table~\ref{table:coop_results} montre les résultats obtenus par les
    quelques stratégies que nous avions à notre disposition pour une durée de
    $100$ tours\footnote{Cette table peut être générée par le script matlab
    \textit{comp\_tests.m} fourni dans l'archive.}.

    Todo: mettre à jour avec les dernières valeurs quand on aura fini et mettre
    les résultats pour un autre nombre de tours.
    \begin{table}[h]
      \centering
      \begin{tabular}{|c||c|c|c|}
        \hline
        Stratégie      & Gain minimal & Gain moyen & Gain maximum \\\hline\hline
        Coopératif     & $110.75$     & $114.02$   & $125.156$    \\\hline
        Non coopératif & $83.25$      & $96.16$    & $109.13$     \\\hline
        Stackelberg    & $54.42$      & $64.67$    & $72.75$      \\\hline
        Pénalise       & $0$          & $44.98687$ & $109.01$     \\\hline
      \end{tabular}
      \caption{Résultats des différentes stratégies}
      \label{table:coop_results}
    \end{table}
    
\section{Annexes}
  \lstinputlisting[label=lst:mmmttk, language=matlab, caption=Statégie Stackelberg sur la moyenne]{duopole/mmmttk.m}
  \lstinputlisting[label=lst:mmmttkv2, language=matlab, caption=Statégie Stackelberg sur la moyenne (variante)]{duopole/mmmttkv2.m}
  \lstinputlisting[label=lst:penalise, language=matlab, caption=Statégie pénalisante]{duopole/penalise.m}
  \lstinputlisting[label=lst:penaliseviolent, language=matlab, caption=Statégie pénalisante (variante)]{duopole/penalise_violent.m}
  
\end{document}
