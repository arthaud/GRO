\begin{algorithm}
  \KwIn{un graphe}
  \KwOut{%
    la liste des cycles ou chaines eulériens selon si le graphe est eulérien ou
    semi-eulérien ou la liste vide sinon
  }
  \Begin{%
    \tcc{Construction de la matrice latine du graphe}
    construire une matrice à n lignes et n colonnes\;
    remplir la matrice de listes vides\;
    \For{chaque nœud du graphe}{%
      \For{chaque arête sortant de ce nœud}{%
        ajouter la liste [nœud de départ, nœud d'arrivée] à la case de
        la matrice correspondante\;
      }
    }
    n $\leftarrow$ le nombre d'arêtes total du graphe\;
    calculer la puissance $(n-1)$ième de la matrice\;

    \For{chaque coefficient de cette matrice}{%
      \If{le coefficient n'est pas nul}{%
        concaténer ce coefficient à la variable de retour\;
      }
    }
  }
  \caption{Méthode de la matrice latine}
  \label{alg:meth_mat_lat}
\end{algorithm}

\begin{algorithm}
  \KwIn{$A$ et $B$ deux matrices latines}
  \KwOut{le produit de ces deux matrices}
  \Begin{%
    construire la matrice de retour à n lignes et n colonnes\;
    initialiser chaque coefficient de cette matrice à la liste vide\;

    \For{chaque coefficient de la matrice de retour}{%
      \For{$k$ allant de $1$ jusqu'à $n$}{%
        calculer les chaines produits entre a(i,k) et b(k,j)\;
        ajouter au coefficient de la matrice ces chaines\;
      }
    }
  }
  \caption{Produit matriciel}
  \label{alg:prod_mat}
\end{algorithm}

\begin{algorithm}
  \KwIn{liste\_1 et liste\_2 deux listes de chaines}
  \KwOut{une liste de chaines}
  \Begin{%
    créer une liste de chaine vide (liste de retour)\;
    \For{$i$ dans liste\_1}{%
      \For{$j$ dans liste\_2}{%
        construire la chaine résultante de la concaténation de $i$ et $j$
        (en enlevant le nœud présent deux fois)\;
        construire un ensemble de chaine vide\;
        \For{$k$ allant de $1$ à la longueur de la chaine construit}{%
          construire la chaine élémentaire menant du nœud $k$ au nœud $k+1$\;
          \If{cette chaine n'est pas dans l'ensemble}{%
            ajouter cette chaine dans l'ensemble\;
          }
          \Else{%
            rendre la chaine nulle\;
            sortir de la boucle\;
          }
        }
      }
      \Si{le chaine n'est pas nulle}{%
        concaténer la chaine trouvée à la liste de retour\;
      }
    }
  }
  \caption{Produit entre listes de chaines (coefficients de matrices
  latines)}
  \label{alg:prod_chaine}
\end{algorithm}

\begin{algorithm}
  \KwIn{un graphe, un sommet de départ optionnel node\_from et un ensemble
  (éventuellement vide) de nœuds déjà parcouru nodes\_done}
  \KwOut{une chaine hamiltonienne sous la forme d'une liste ordonnée de
  sommets, ou None s'il n'en existe pas}

  \Begin{%
    \If{node\_from n'a pas été fourni}{%
          node\_from $\leftarrow$ un nœud du graphe\;
    }
    ajouter node\_from à nodes\_dones\;

    \If{cardinal(node\_from) == ordre(graphe)}{%
      \Return{[node\_from]}\;
    }

    \For{chaque arête dans le graphe}{%
      autre $\leftarrow$ le sommet opposé à node\_from par rapport à cette
      arête\;
      \If{autre dans nodes\_done}{%
        passer à la prochaine arête\;
      }
      appeler la fonction récursivement avec le graphe, node\_from et
      nodes\_dones comme paramètres\;
      \If{la liste retournée est non-vide}{%
          y ajouter node\_from au début et la retourner\;
      }
    }
  }
  \caption{Recherche de chaine hamiltonienne}
  \label{alg:hamil}
\end{algorithm}


\begin{algorithm}
  \KwIn{un graphe complet g}
  \KwOut{(cout, cycle) où cycle est un cycle hamiltonien construit selon la
  méthode du plus proche voisin et cout son cout associé sous forme de
  liste de sommets}
  \Begin{%
    cout = 0\;
    cycle = une liste composée d'un sommet de g au hasard\;

    \While{il reste des sommets}{%
      \tcc{On ajoute au cycle le sommet suivant}
      plus\_proche = sommet de g sur lequel on est pas encore passé le
      plus proche du dernier sommet du cycle\;

      cout += cout de plus\_proche au dernier sommet du cycle construit\;
      ajouter plus\_proche au cycle\;
    }

    \tcc{On ferme le cycle}
    cout += cout du dernier au premier sommet de cycle\;
    ajouter le premier sommet de la chaine à cycle\;

    \Return{(cout, cycle)}
  }
  \label{alg:voisin}
  \caption{Plus proche voisin}
\end{algorithm}

\begin{algorithm}
  \KwIn{un cycle hamiltonien (liste de sommets) et son cout}
  \KwOut{un cycle hamiltonien et son cout inférieur ou égal au coup
  d'entrée}

  \Begin{%
    \For{chaque couple de sommets (a, b) dans le cycle}{%
       $\begin{aligned}
\text{nouveau\_cout} & \leftarrow
\text{cout}\\
&- \text{cout de a à son successeur dans le cycle}\\
&- \text{cout de b à son successeur dans le cycle}\\
&+ \text{cout de a à b}\\
&+ \text{cout du successeur de a au successeur de b dans le
cycle}\\
\end{aligned}$\;

      \If{nouveau\_cout < cout}{%
        cout = nouveau\_cout
        cycle = cycle crée en échangeant a et b dans cycle\;
      }
    }
    \Return{(cout, cycle)}
  }
  \caption{2-opt}
  \label{alg:2opt}
\end{algorithm}

\begin{algorithm}
  \KwIn{Un graphe non-orienté et connexe g}
  \KwOut{le cycle le plus court permettant de visiter toutes les arêtes de
  g}

  \Begin{%
    \tcc{Création du graphe partiel}
    \For{chaque sommet de g}{%
      \If{le sommet est de degré impair}{%
        créer le sommet dans graphe\_partiel\;
      }
    }
    \For{chaque arête de g}{%
      \If{ses 2 sommets sont dans graphe\_partiel}{%
        créer la même arête dans graphe\_partiel\;
      }
    }

    \tcc{Transformation en clique}
    \For{chaque couple de sommet (s1, s2) dans graphe\_partiel}{%
      \If{il n'y a pas d'arête reliant s1 et s2}{%
        $(cout, chaine) = dijkstra(s1, s2)$\;
        créer l'arête reliant s1 et s2 dans graphe\_partiel, de cout cout\;
      }
    }

    $couplage, cout = aux(ensemble vide, ensemble vide, 0)$\;

    \For{chaque arête dans couplage}{%
      $(s1, s2) =$ sommets reliés par arête dans g\;
      $(cout, chaine) = dijkstra(s1, s2)$\;
      \For{chaque arête dans chaine}{%
        doubler cette arête dans g\;
      }
    }

    \Return{le cycle eulérien de g}
  }
  \label{alg:postier_chinois}
  \caption{Algorithme du postier chinois}
\end{algorithm}

\begin{algorithm}
  \KwIn{arêtes, sommets\_visités, cout}
  \Begin{%
    \If{sommets\_visités contient tous les sommets de graphe\_partiel}{%
      \Return{(arêtes, cout)}
    }
    \Else{%
      $meilleur\_couplage = Vide$\;
      $meilleur\_cout = 0$\;
      \For{chaque arête de graphe\_partiel}{%
        \If{les 2 sommets de l'arête ne sont pas dans sommets\_visités}{%
          arêtes\_copie = copie de arêtes\;
          sommets\_visités\_copie = copie de sommets\_visités\;

          ajouter arête dans arêtes\_copie\;
          couplage, cout = aux(arêtes\_copie, sommets\_visités\_copie, cout +
          cout de arête)\;

          \If{meilleur\_couplage = Vide ou meilleur\_cout > cout}{%
            meilleur\_couplage = couplage\;
            meilleur\_cout = cout\;
          }
        }
      }
      \Return{(meilleur\_couplage, meilleur\_cout)}
    }
  }
  \label{alg:couplage}
  \caption{Recherche du couplage parfait de cout minimum par bruteforce}
\end{algorithm}

\begin{algorithm}
  \KwIn{une liste d'objets}
  \KwIn{une masse maximale autorisée}
  \KwOut{la valeur maximale qu'il est possible d'atteindre}
  \Begin{%
    ligne\_courante = liste composée de (masse\_max+1) zéros\;
    ligne\_prec     = liste composée de (masse\_max+1) zéros\;
    \For{chaque objet obj de la liste d'entrée}{%
      \For{m variant de 0 à masse\_max}{%
        \If{masse(obj) $\leq$ m}{%
          ligne\_courante[m] $\leftarrow$ max(ligne\_prec[m], ligne\_prec\_line[m-masse(obj)] + prix(obj))\;
        }
      }
      ligne\_prec $\leftarrow$ ligne\_courante\;
    }
    \Return{ligne\_courante[masse\_max]}
  }
  \caption{Algorithme de résolution exacte du problème du sac à dos}
  \label{alg:sacados}
\end{algorithm}

\begin{algorithm}
  \KwIn{Matrice (tableau canonique)}
  \KwOut{Gain maximal, quantités de
    chaque produit à faire, quantité des ressources restantes}
  \Begin{%
    base = indices des variables de base de la matrice\;

    \While{il reste un coefficient strictement positif dans les bénéfices}{%
      à\_ajouter = indice du produit qui rapporte le plus\;
      à\_retirer = $\underset{i}{\text{argmin}}
      \frac{stock(i)}{matrice(\text{à\_ajouter},i)}$\;

      base(à\_retirer) = à\_ajouter\;

      matrice(à\_retirer,:) = matrice(à\_retirer,:) /
      matrice(à\_retirer,à\_ajouter)\; 
      \For{$i$ allant de $1$ au nombre de lignes}{%
        \If{$i \neq $à\_ajouter}{%
          matrice($i$,:) = matrice(à\_retirer,:) $\times$ matrice($i$, à\_ajouter) /
          matrice(à\_ajouter, à\_retirer)\;
        }
      }
    }

    \For{$n$ allant de 0 au nombre de variable hors base -1}{%
      à\_produire(base($n$)) = matrice($n+1$,stock)\;
    }

    \For{$n$ allant du nombre de variables hors base au nombre de variables
    total}{%
      restes(base($n$)) = matrice($n+1$,stock)\;
    }

    \Return{(-\text{matrice(0, stock), à\_produire, restes})}
  }
  \label{alg:simpl}
  \caption{Algorithme du simplexe}
\end{algorithm}

\begin{algorithm}
  \KwIn{deux sommets s1 et s2}
  %\Require{il existe une chaine entre les deux sommets}
  \KwOut{le plus court chemin entre s1 et s2}
  \Begin{%
    \For{chaque sommet du graphe}{%
      sommet.parcouru $\leftarrow \infty$\;
      sommet.precedent $\leftarrow 0$\;
    }
    s1.parcouru $\leftarrow 0$\;
    non\_visités $\leftarrow$ ensemble des sommets du graphe\;

    \While{non\_visités est non vide}{%
      s $\leftarrow$ le sommet de non\_visités avec parcouru minimum\;
      supprimer s de non\_visités\;

      \For{chaque sommet f fils de s}{%
        \If{f.parcouru > s.parcouru + poids de l'arc entre s et f}{%
          f.parcouru $\leftarrow$ s.parcouru + poids de l'arc entre s et f\;
          f.precedent $\leftarrow$ s\;
          ajouter f dans non\_visités\;
        }
      }
    }

    chaine $\leftarrow$ liste vide\;
    s = s2\;
    \While{s $\neq$ s1}{%
      chaine.ajouter(s)\;
      s $\leftarrow$ s.precedent\;
    }

    chaine.ajouter(s1)\;
    \Return{chaine}
  }
  \caption{Algorithme de Dijkstra}
  \label{alg:dijkstra}
\end{algorithm}

