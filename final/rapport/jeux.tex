\subsection{Shifumi}
    Une stratégie simple et efficace à laquelle on pourrait penser pour gagner
    au Shifumi serait de jouer de manière aléatoire.

    Et en effet, il s'avère que si les deux joueurs jouent de manière équiprobable,
    on a affaire à un équilibre de Nash~: aucun changement de stratégie de la part
    d'un joueur ne pourra lui permettre d'augmenter ses chances de gagner.

    De plus, si un adversaire ne joue pas de manière aléatoire (ou augmente la
    probabilité de jouer un certain élément), alors on pourra prévoir ce qu'il
    va jouer et donc trouver une stratégie qui pourra le battre. Les humains
    étant très mauvais pour jouer de manière aléatoire, il est assez facile
    d'écrire une stratégie permettant de les battre.

  \subsubsection{Stratégie développée}
    Afin de démontrer qu'un adversaire ne jouant pas aléatoirement est facile
    à battre, nous avons développé une stratégie qui se base sur des chaînes
    de Markov~: en se basant sur les derniers éléments joués, elle regarde
    dans l'historique pour voir l'élément qui était joué le plus souvent par
    l'adversaire après les derniers coups joués.

    Cette stratégie s'avère vraiment efficace contre un joueur humain.
    Toutefois, elle est prévisible~: si on sait qu'on a affaire à une telle
    stratégie, on peut jouer de manière à la battre.

    C'est pour cela qu'une stratégie aléatoire est la seule pouvant maximiser
    nos gains dans le pire des cas.

  \subsubsection{Variantes}
    Toutes les variantes du Shifumi qui consistent à rajouter des éléments
    pour obtenir un nombre d'éléments pair (par exemple
    pierre/papier/ciseaux/puits) vont créer un déséquilibre, car un élément
    sera moins efficace contre les autres. L'équilibre de Nash du jeu va
    alors consister à ne jamais jouer cet élément.

    Si le nombre d'éléments est impair, alors le jeu pourra être équilibré,
    comme un Shifumi classique.

