\documentclass{scrartcl}
\usepackage[utf8]{inputenc}
\usepackage[frenchb]{babel}
\usepackage{lmodern}
\usepackage[T1]{fontenc}
\usepackage{hyperref}
\usepackage{verbatim}
\usepackage{listings}
\usepackage{graphicx}


\usepackage{color}
\definecolor{deepblue}{rgb}{0,0,0.5}
\definecolor{deepred}{rgb}{0.6,0,0}
\definecolor{deepgreen}{rgb}{0,0.5,0}

\newcommand{\brokencell}[2][c]{\begin{tabular}[#1]{@{}c@{}}#2\end{tabular}}

\lstset{frame=single, breaklines=true,
          breakatwhitespace=true, basicstyle=\scriptsize,
          showstringspaces=false, escapeinside={(*}{*)},
          keywordstyle=\color{deepblue},
          stringstyle=\color{deepred},
          commentstyle=\color{deepgreen},
          literate=
                   {é}{{\'e}}1{É}{{\'E}}1
                   {è}{{\`e}}1{È}{{\`E}}1
                   {ê}{{\^e}}1{Ê}{{\^E}}1
                   {à}{{\`a}}1{À}{{\`A}}1
                   {ù}{{\`u}}1{Ù}{{\`U}}1
                   {û}{{\^u}}1{Û}{{\^U}}1
                   {ô}{{\^o}}1{Ô}{{\^O}}1
                   {ó}{{\'o}}1{Ó}{{\'O}}1
                   {ç}{{\c c}}1{Ç}{{\c C}}1
                   {œ}{{\oe}}1{Œ}{{\OE}}1
        }

\begin{document}
\title{Rapport du projet de programmation linéaire}
\author{Maxence Ahlouche \and Maxime Arthaud \and Korantin Auguste
          \and Martin Carton \and Thomas Forgione \and Thomas Wagner}
\date{21 octobre 2013}
\maketitle
\tableofcontents
\newpage

\section{Présentation de l'équipe}
  Cette équipe a été menée par Maxence Ahlouche, assisté de son Responsable
  Qualité Thomas Wagner. Les autres membres de l'équipe sont Martin Carton,
  Thomas Forgione, Maxime Arthaud, et Korantin Auguste.

  %todo: absences
  
\section{Problème du sac à dos}
  \begin{center}\includegraphics[width=120pt]{sac_a_dos.jpg}\end{center}

  \subsection{Résolution exacte}
    Nous avons implémenté un algorithme de programmation dynamique, qui permet
    de résoudre le problème du sac à dos.  Toutefois, il fonctionne uniquement
    si les poids des objets sont des entiers.

    Sa complexité en temps est en $O(nW)$ et celle en mémoire en $O(W)$, avec
    $n$ le nombre d'objets et $W$ le poids maximum du sac.

    Nous l'avons testé sur plusieurs instances du problème (jusqu'à X objets et
    un poids maximal de X), et l'algorithme s'exécute toujours en moins d'une
    seconde. %todo: faire les tests :D

  \subsection{Résolution approchée}
    Nous avons aussi implémenté l'algorithme glouton: celui-ci consiste à
    choisir les \og meilleurs \fg{} objets jusqu'à que la masse maximale soit
    dépassée.  Le critère déterminant quels sont les meilleurs objets peuvent
    être la masse faible, le prix élevé, ou le rapport prix/masse élevé.

    Cet algorithme est beaucoup plus rapide que le précédent, mais n'est qu'un
    algorithme approché. La table~\ref{table:greedy} montre les résultats
    obtenus.

    %todo: ajouter des tests avec un ``range of coeficients'' plus élevé pour
    %avoir des résultats plus intéressant avec le ratio prix/masse
    \begin{table}[h]
      \makebox[\textwidth]{%
      \centering
      \begin{tabular}{| c | c | c | c | c | c |}
      \hline
        \brokencell{Paramètres du générateur/\\masse maximale autorisée}
      & \brokencell{Résultat\\optimum}
      & \brokencell{Prix le\\plus élevé}
      & \brokencell{Masse la\\plus faible}
      & \brokencell{Meilleur ratio\\prix/masse}\\
      \hline
      500 25 1 1 1000/500& $2016$ & $1125/44.2\%$ & $1725/14.4\%$ & $2016/0\%$ \\
      5000 25 1 1 1000/500& $5540$ & $1175/79\%$ & $4577/17.4\%$ & $5540/0\%$ \\
      50000 25 1 1 1000/500& $11195$ & $1175/90\%$ & $6684/40.3\%$ & $11195/0\%$ \\
      50000 25 1 1 1000/5000& $11195$ & $1175/90\%$ & $6684/40.3\%$ & $11195/0\%$ \\
      \hline  
      \end{tabular}
      }
      \caption{Résultats de l'algorithme glouton}
      \label{table:greedy}
    \end{table}

\section{Annexe}
  \lstlistoflistings
  \lstinputlisting[language=python, caption=Codes relatifs au problème du sac à dos]{sacados.py}

\end{document}
