\documentclass{scrartcl}
\usepackage[utf8]{inputenc}
\usepackage[frenchb]{babel}
\usepackage{lmodern}
\usepackage[T1]{fontenc}
\usepackage{hyperref}
\usepackage{verbatim}
\usepackage{listings}
\usepackage{graphicx}


\usepackage{color}
\definecolor{deepblue}{rgb}{0,0,0.5}
\definecolor{deepred}{rgb}{0.6,0,0}
\definecolor{deepgreen}{rgb}{0,0.5,0}

\newcommand{\brokencell}[2][c]{\begin{tabular}[#1]{@{}c@{}}#2\end{tabular}}

\lstset{frame=single, breaklines=true,
          breakatwhitespace=true, basicstyle=\scriptsize,
          showstringspaces=false, escapeinside={(*}{*)},
          keywordstyle=\color{deepblue},
          stringstyle=\color{deepred},
          commentstyle=\color{deepgreen},
          literate=
                   {é}{{\'e}}1{É}{{\'E}}1
                   {è}{{\`e}}1{È}{{\`E}}1
                   {ê}{{\^e}}1{Ê}{{\^E}}1
                   {à}{{\`a}}1{À}{{\`A}}1
                   {ù}{{\`u}}1{Ù}{{\`U}}1
                   {û}{{\^u}}1{Û}{{\^U}}1
                   {ô}{{\^o}}1{Ô}{{\^O}}1
                   {ó}{{\'o}}1{Ó}{{\'O}}1
                   {ç}{{\c c}}1{Ç}{{\c C}}1
                   {œ}{{\oe}}1{Œ}{{\OE}}1
        }

\begin{document}
\title{Rapport du projet de programmation linéaire}
\author{Maxence Ahlouche \and Maxime Arthaud \and Korantin Auguste
          \and Martin Carton \and Thomas Forgione \and Thomas Wagner}
\date{?}
\maketitle
\newpage
\tableofcontents
\newpage

\section{Problème du sac à dos}

\begin{center}\includegraphics[width=120pt]{sac_a_dos.jpg}\end{center}

Nous avons implémenté un algorithme de programmation dynamique, qui permet de résoudre le problème du sac à dos.
Toutefois, il fonctionne uniquement si les poids des objets sont des entiers.

Sa complexité en temps et en mémoire est en $O(nW)$, avec $n$ le nombre d'objets et $W$ le poids maximum du sac.

Nous l'avons testé sur plusieurs instances du problème (jusqu'à X objets et un poids maximal de X), et l'algorithme s'exécute toujours en moins d'une seconde.

\lstinputlisting{sacados.py}

Nous aurions aussi pu faire un algorithme glouton, en triant les objets par rapport prix / poids.

\end{document}
